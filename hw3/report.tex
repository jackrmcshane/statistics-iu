% Options for packages loaded elsewhere
\PassOptionsToPackage{unicode}{hyperref}
\PassOptionsToPackage{hyphens}{url}
%
\documentclass[
]{article}
\usepackage{amsmath,amssymb}
\usepackage{lmodern}
\usepackage{iftex}
\ifPDFTeX
  \usepackage[T1]{fontenc}
  \usepackage[utf8]{inputenc}
  \usepackage{textcomp} % provide euro and other symbols
\else % if luatex or xetex
  \usepackage{unicode-math}
  \defaultfontfeatures{Scale=MatchLowercase}
  \defaultfontfeatures[\rmfamily]{Ligatures=TeX,Scale=1}
\fi
% Use upquote if available, for straight quotes in verbatim environments
\IfFileExists{upquote.sty}{\usepackage{upquote}}{}
\IfFileExists{microtype.sty}{% use microtype if available
  \usepackage[]{microtype}
  \UseMicrotypeSet[protrusion]{basicmath} % disable protrusion for tt fonts
}{}
\makeatletter
\@ifundefined{KOMAClassName}{% if non-KOMA class
  \IfFileExists{parskip.sty}{%
    \usepackage{parskip}
  }{% else
    \setlength{\parindent}{0pt}
    \setlength{\parskip}{6pt plus 2pt minus 1pt}}
}{% if KOMA class
  \KOMAoptions{parskip=half}}
\makeatother
\usepackage{xcolor}
\IfFileExists{xurl.sty}{\usepackage{xurl}}{} % add URL line breaks if available
\IfFileExists{bookmark.sty}{\usepackage{bookmark}}{\usepackage{hyperref}}
\hypersetup{
  pdftitle={S520 Homework 3},
  pdfauthor={Jack McShane},
  hidelinks,
  pdfcreator={LaTeX via pandoc}}
\urlstyle{same} % disable monospaced font for URLs
\usepackage[margin=1in]{geometry}
\usepackage{color}
\usepackage{fancyvrb}
\newcommand{\VerbBar}{|}
\newcommand{\VERB}{\Verb[commandchars=\\\{\}]}
\DefineVerbatimEnvironment{Highlighting}{Verbatim}{commandchars=\\\{\}}
% Add ',fontsize=\small' for more characters per line
\usepackage{framed}
\definecolor{shadecolor}{RGB}{248,248,248}
\newenvironment{Shaded}{\begin{snugshade}}{\end{snugshade}}
\newcommand{\AlertTok}[1]{\textcolor[rgb]{0.94,0.16,0.16}{#1}}
\newcommand{\AnnotationTok}[1]{\textcolor[rgb]{0.56,0.35,0.01}{\textbf{\textit{#1}}}}
\newcommand{\AttributeTok}[1]{\textcolor[rgb]{0.77,0.63,0.00}{#1}}
\newcommand{\BaseNTok}[1]{\textcolor[rgb]{0.00,0.00,0.81}{#1}}
\newcommand{\BuiltInTok}[1]{#1}
\newcommand{\CharTok}[1]{\textcolor[rgb]{0.31,0.60,0.02}{#1}}
\newcommand{\CommentTok}[1]{\textcolor[rgb]{0.56,0.35,0.01}{\textit{#1}}}
\newcommand{\CommentVarTok}[1]{\textcolor[rgb]{0.56,0.35,0.01}{\textbf{\textit{#1}}}}
\newcommand{\ConstantTok}[1]{\textcolor[rgb]{0.00,0.00,0.00}{#1}}
\newcommand{\ControlFlowTok}[1]{\textcolor[rgb]{0.13,0.29,0.53}{\textbf{#1}}}
\newcommand{\DataTypeTok}[1]{\textcolor[rgb]{0.13,0.29,0.53}{#1}}
\newcommand{\DecValTok}[1]{\textcolor[rgb]{0.00,0.00,0.81}{#1}}
\newcommand{\DocumentationTok}[1]{\textcolor[rgb]{0.56,0.35,0.01}{\textbf{\textit{#1}}}}
\newcommand{\ErrorTok}[1]{\textcolor[rgb]{0.64,0.00,0.00}{\textbf{#1}}}
\newcommand{\ExtensionTok}[1]{#1}
\newcommand{\FloatTok}[1]{\textcolor[rgb]{0.00,0.00,0.81}{#1}}
\newcommand{\FunctionTok}[1]{\textcolor[rgb]{0.00,0.00,0.00}{#1}}
\newcommand{\ImportTok}[1]{#1}
\newcommand{\InformationTok}[1]{\textcolor[rgb]{0.56,0.35,0.01}{\textbf{\textit{#1}}}}
\newcommand{\KeywordTok}[1]{\textcolor[rgb]{0.13,0.29,0.53}{\textbf{#1}}}
\newcommand{\NormalTok}[1]{#1}
\newcommand{\OperatorTok}[1]{\textcolor[rgb]{0.81,0.36,0.00}{\textbf{#1}}}
\newcommand{\OtherTok}[1]{\textcolor[rgb]{0.56,0.35,0.01}{#1}}
\newcommand{\PreprocessorTok}[1]{\textcolor[rgb]{0.56,0.35,0.01}{\textit{#1}}}
\newcommand{\RegionMarkerTok}[1]{#1}
\newcommand{\SpecialCharTok}[1]{\textcolor[rgb]{0.00,0.00,0.00}{#1}}
\newcommand{\SpecialStringTok}[1]{\textcolor[rgb]{0.31,0.60,0.02}{#1}}
\newcommand{\StringTok}[1]{\textcolor[rgb]{0.31,0.60,0.02}{#1}}
\newcommand{\VariableTok}[1]{\textcolor[rgb]{0.00,0.00,0.00}{#1}}
\newcommand{\VerbatimStringTok}[1]{\textcolor[rgb]{0.31,0.60,0.02}{#1}}
\newcommand{\WarningTok}[1]{\textcolor[rgb]{0.56,0.35,0.01}{\textbf{\textit{#1}}}}
\usepackage{graphicx}
\makeatletter
\def\maxwidth{\ifdim\Gin@nat@width>\linewidth\linewidth\else\Gin@nat@width\fi}
\def\maxheight{\ifdim\Gin@nat@height>\textheight\textheight\else\Gin@nat@height\fi}
\makeatother
% Scale images if necessary, so that they will not overflow the page
% margins by default, and it is still possible to overwrite the defaults
% using explicit options in \includegraphics[width, height, ...]{}
\setkeys{Gin}{width=\maxwidth,height=\maxheight,keepaspectratio}
% Set default figure placement to htbp
\makeatletter
\def\fps@figure{htbp}
\makeatother
\setlength{\emergencystretch}{3em} % prevent overfull lines
\providecommand{\tightlist}{%
  \setlength{\itemsep}{0pt}\setlength{\parskip}{0pt}}
\setcounter{secnumdepth}{-\maxdimen} % remove section numbering
\ifLuaTeX
  \usepackage{selnolig}  % disable illegal ligatures
\fi

\title{S520 Homework 3}
\author{Jack McShane}
\date{2022-02-19}

\begin{document}
\maketitle

\hypertarget{problem-1}{%
\subsection{Problem 1}\label{problem-1}}

\begin{itemize}
\item
  \begin{enumerate}
  \def\labelenumi{\alph{enumi})}
  \setcounter{enumi}{2}
  \tightlist
  \item
    Determine the expected value of X.
  \end{enumerate}
\end{itemize}

\begin{center}
    The expected value for a random variable can be calculated by taking a
    weighted sum where each possible value of the variable is weighted according
    to its probability. This can be done in R as so:
\end{center}

\begin{Shaded}
\begin{Highlighting}[]
\NormalTok{vals }\OtherTok{\textless{}{-}} \FunctionTok{c}\NormalTok{(}\DecValTok{1}\NormalTok{, }\DecValTok{3}\NormalTok{, }\DecValTok{4}\NormalTok{, }\DecValTok{6}\NormalTok{)}
\NormalTok{probs }\OtherTok{\textless{}{-}} \FunctionTok{c}\NormalTok{(.}\DecValTok{1}\NormalTok{, .}\DecValTok{4}\NormalTok{, .}\DecValTok{4}\NormalTok{, .}\DecValTok{1}\NormalTok{)}
\NormalTok{expectation }\OtherTok{\textless{}{-}} \FunctionTok{sum}\NormalTok{(vals}\SpecialCharTok{*}\NormalTok{probs)}
\end{Highlighting}
\end{Shaded}

\begin{verbatim}
## [1] "Expected value:  3.5"
\end{verbatim}

\hfill\break

\begin{itemize}
\item
  \begin{enumerate}
  \def\labelenumi{\alph{enumi})}
  \setcounter{enumi}{3}
  \tightlist
  \item
    Determine the variance of X.
  \end{enumerate}
\end{itemize}

\begin{center}
    The variance of a random variable measures how much observed values differ
    from the mean or expected value of the variable. It can be calculated using
    R in this manner:
\end{center}

\begin{Shaded}
\begin{Highlighting}[]
\NormalTok{squared\_diff }\OtherTok{\textless{}{-}}\NormalTok{ (vals }\SpecialCharTok{{-}}\NormalTok{ expectation)}\SpecialCharTok{\^{}}\DecValTok{2}
\NormalTok{variance }\OtherTok{\textless{}{-}} \FunctionTok{sum}\NormalTok{(squared\_diff }\SpecialCharTok{*}\NormalTok{ probs)}
\end{Highlighting}
\end{Shaded}

\begin{verbatim}
## [1] "Variance of X:  1.45"
\end{verbatim}

\hfill\break

\begin{itemize}
\item
  \begin{enumerate}
  \def\labelenumi{\alph{enumi})}
  \setcounter{enumi}{4}
  \tightlist
  \item
    Determine the standard deviation of X.
  \end{enumerate}
\end{itemize}

\begin{center}
    The standard deviation of a random variable's distribution indicates how
    spread values of the random variable are from the mean. It can be calculated
    as so:
\end{center}

\begin{Shaded}
\begin{Highlighting}[]
\NormalTok{stddev }\OtherTok{\textless{}{-}} \FunctionTok{sqrt}\NormalTok{(variance)}
\end{Highlighting}
\end{Shaded}

\begin{verbatim}
## [1] "Standard Deviation of X:  1.20415945787923"
\end{verbatim}

\hfill\break

\hypertarget{problem-2}{%
\subsection{Problem 2}\label{problem-2}}

\begin{itemize}
\item
  \begin{enumerate}
  \def\labelenumi{\alph{enumi})}
  \tightlist
  \item
    Determine the pmf of X.
  \end{enumerate}
\end{itemize}

\begin{center}
    The pmf can be calculated as shown below. The get_probs function calculates
    the probability of each value of X according to the equation given in the
    problem.
\end{center}

\begin{Shaded}
\begin{Highlighting}[]
\NormalTok{get\_probs }\OtherTok{=} \ControlFlowTok{function}\NormalTok{(vals) \{}
\NormalTok{  probs }\OtherTok{\textless{}{-}} \FunctionTok{vector}\NormalTok{()}
  
  \ControlFlowTok{for}\NormalTok{ (i }\ControlFlowTok{in} \FunctionTok{seq\_along}\NormalTok{(vals))\{}
    
    \ControlFlowTok{if}\NormalTok{ (vals[i] }\SpecialCharTok{==} \DecValTok{6}\NormalTok{)\{}
\NormalTok{      probs[i] }\OtherTok{\textless{}{-}} \DecValTok{0}
\NormalTok{    \} }\ControlFlowTok{else}\NormalTok{ \{}
\NormalTok{      probs[i] }\OtherTok{\textless{}{-}}\NormalTok{ (}\DecValTok{7} \SpecialCharTok{{-}}\NormalTok{ vals[i]) }\SpecialCharTok{/} \DecValTok{20}
\NormalTok{    \}}
    
\NormalTok{  \}}
  \FunctionTok{return}\NormalTok{(probs)}
\NormalTok{\}}

\NormalTok{vals }\OtherTok{=} \FunctionTok{c}\NormalTok{(}\DecValTok{1}\NormalTok{, }\DecValTok{2}\NormalTok{, }\DecValTok{3}\NormalTok{, }\DecValTok{4}\NormalTok{, }\DecValTok{5}\NormalTok{, }\DecValTok{6}\NormalTok{)}
\NormalTok{probs }\OtherTok{\textless{}{-}} \FunctionTok{get\_probs}\NormalTok{(vals)}
\end{Highlighting}
\end{Shaded}

\begin{verbatim}
## [1] "Probabilities:  0.3"  "Probabilities:  0.25" "Probabilities:  0.2" 
## [4] "Probabilities:  0.15" "Probabilities:  0.1"  "Probabilities:  0"
\end{verbatim}

\includegraphics{report_files/figure-latex/unnamed-chunk-8-1.pdf}

\hfill\break

\begin{itemize}
\item
  \begin{enumerate}
  \def\labelenumi{\alph{enumi})}
  \setcounter{enumi}{1}
  \tightlist
  \item
    Determine the cdf of X.
  \end{enumerate}
\end{itemize}

\begin{Shaded}
\begin{Highlighting}[]
\NormalTok{get\_cdf }\OtherTok{=} \ControlFlowTok{function}\NormalTok{(pmf\_probs) \{}
\NormalTok{  cdf\_probs }\OtherTok{\textless{}{-}} \FunctionTok{vector}\NormalTok{()}
  
  \ControlFlowTok{for}\NormalTok{ (i }\ControlFlowTok{in} \FunctionTok{seq\_along}\NormalTok{(pmf\_probs))\{}
    
    \ControlFlowTok{if}\NormalTok{ (i }\SpecialCharTok{==} \DecValTok{1}\NormalTok{)\{}
\NormalTok{      cdf\_probs[i] }\OtherTok{\textless{}{-}}\NormalTok{ pmf\_probs[i]}
\NormalTok{    \} }\ControlFlowTok{else}\NormalTok{ \{}
\NormalTok{      cdf\_probs[i] }\OtherTok{\textless{}{-}}\NormalTok{ pmf\_probs[i] }\SpecialCharTok{+}\NormalTok{ cdf\_probs[i }\SpecialCharTok{{-}} \DecValTok{1}\NormalTok{]}
\NormalTok{    \}}
    
\NormalTok{  \}}
  
  \FunctionTok{return}\NormalTok{(cdf\_probs)}
\NormalTok{\}}

\NormalTok{cdf\_probs }\OtherTok{\textless{}{-}} \FunctionTok{get\_cdf}\NormalTok{(probs)}
\NormalTok{frame }\OtherTok{\textless{}{-}} \FunctionTok{data.frame}\NormalTok{(vals, cdf\_probs)}

\FunctionTok{ggplot}\NormalTok{(}\AttributeTok{data =}\NormalTok{ frame,}
        \AttributeTok{mapping =} \FunctionTok{aes}\NormalTok{(}\AttributeTok{x =}\NormalTok{ vals, }\AttributeTok{y =}\NormalTok{ cdf\_probs)) }\SpecialCharTok{+} \FunctionTok{geom\_col}\NormalTok{(}\FunctionTok{aes}\NormalTok{(}\AttributeTok{x=}\NormalTok{vals, }\AttributeTok{y=}\NormalTok{cdf\_probs))}
\end{Highlighting}
\end{Shaded}

\includegraphics{report_files/figure-latex/unnamed-chunk-9-1.pdf}

\begin{Shaded}
\begin{Highlighting}[]
\CommentTok{\# plot(vals, cdf\_probs, type=\textquotesingle{}h\textquotesingle{})}
\end{Highlighting}
\end{Shaded}

\hfill\break

\begin{itemize}
\item
  \begin{enumerate}
  \def\labelenumi{\alph{enumi})}
  \setcounter{enumi}{2}
  \tightlist
  \item
    Determine the expected value of X.
  \end{enumerate}
\end{itemize}

\begin{Shaded}
\begin{Highlighting}[]
\NormalTok{expectation }\OtherTok{\textless{}{-}} \FunctionTok{sum}\NormalTok{(vals}\SpecialCharTok{*}\NormalTok{probs)}
\FunctionTok{print}\NormalTok{(expectation)}
\end{Highlighting}
\end{Shaded}

\begin{verbatim}
## [1] 2.5
\end{verbatim}

\hfill\break

\begin{itemize}
\item
  \begin{enumerate}
  \def\labelenumi{\alph{enumi})}
  \setcounter{enumi}{3}
  \tightlist
  \item
    Determine the variance of X.
  \end{enumerate}
\end{itemize}

\begin{Shaded}
\begin{Highlighting}[]
\NormalTok{squared\_diff }\OtherTok{\textless{}{-}}\NormalTok{ (vals }\SpecialCharTok{{-}}\NormalTok{ expectation)}\SpecialCharTok{\^{}}\DecValTok{2}
\NormalTok{variance }\OtherTok{\textless{}{-}} \FunctionTok{sum}\NormalTok{(squared\_diff }\SpecialCharTok{*}\NormalTok{ probs)}
\FunctionTok{print}\NormalTok{(variance)}
\end{Highlighting}
\end{Shaded}

\begin{verbatim}
## [1] 1.75
\end{verbatim}

\hfill\break

\begin{itemize}
\item
  \begin{enumerate}
  \def\labelenumi{\alph{enumi})}
  \setcounter{enumi}{4}
  \tightlist
  \item
    Determine the standard deviation of X.
  \end{enumerate}
\end{itemize}

\begin{Shaded}
\begin{Highlighting}[]
\NormalTok{stddev }\OtherTok{\textless{}{-}} \FunctionTok{sqrt}\NormalTok{(variance)}
\FunctionTok{print}\NormalTok{(stddev)}
\end{Highlighting}
\end{Shaded}

\begin{verbatim}
## [1] 1.322876
\end{verbatim}

\hfill\break

\hypertarget{problem-3}{%
\subsection{Problem 3}\label{problem-3}}

\begin{itemize}
\item
  \begin{enumerate}
  \def\labelenumi{\alph{enumi})}
  \tightlist
  \item
    Determine the pmf of X.
  \end{enumerate}
\end{itemize}

\begin{Shaded}
\begin{Highlighting}[]
\NormalTok{vals }\OtherTok{=} \FunctionTok{c}\NormalTok{(}\DecValTok{1}\NormalTok{, }\DecValTok{2}\NormalTok{, }\DecValTok{5}\NormalTok{, }\DecValTok{10}\NormalTok{)}
\NormalTok{probs }\OtherTok{\textless{}{-}} \FunctionTok{c}\NormalTok{((}\DecValTok{4}\SpecialCharTok{/}\DecValTok{10}\NormalTok{), (}\DecValTok{1}\SpecialCharTok{/}\DecValTok{10}\NormalTok{), (}\DecValTok{2}\SpecialCharTok{/}\DecValTok{10}\NormalTok{), (}\DecValTok{3}\SpecialCharTok{/}\DecValTok{10}\NormalTok{))}

\FunctionTok{print}\NormalTok{(vals)}
\end{Highlighting}
\end{Shaded}

\begin{verbatim}
## [1]  1  2  5 10
\end{verbatim}

\begin{Shaded}
\begin{Highlighting}[]
\FunctionTok{print}\NormalTok{(probs)}
\end{Highlighting}
\end{Shaded}

\begin{verbatim}
## [1] 0.4 0.1 0.2 0.3
\end{verbatim}

\begin{Shaded}
\begin{Highlighting}[]
\FunctionTok{plot}\NormalTok{(vals, probs, }\AttributeTok{type=}\StringTok{\textquotesingle{}h\textquotesingle{}}\NormalTok{)}
\end{Highlighting}
\end{Shaded}

\includegraphics{report_files/figure-latex/unnamed-chunk-13-1.pdf}\\

\begin{itemize}
\item
  \begin{enumerate}
  \def\labelenumi{\alph{enumi})}
  \setcounter{enumi}{1}
  \tightlist
  \item
    Determine the cdf of X.
  \end{enumerate}
\end{itemize}

\begin{Shaded}
\begin{Highlighting}[]
\NormalTok{cdf\_probs }\OtherTok{=} \FunctionTok{get\_cdf}\NormalTok{(probs)}
\FunctionTok{plot}\NormalTok{(vals, cdf\_probs, }\AttributeTok{type=}\StringTok{\textquotesingle{}h\textquotesingle{}}\NormalTok{)}
\end{Highlighting}
\end{Shaded}

\includegraphics{report_files/figure-latex/unnamed-chunk-14-1.pdf}\\

\begin{itemize}
\item
  \begin{enumerate}
  \def\labelenumi{\alph{enumi})}
  \setcounter{enumi}{2}
  \tightlist
  \item
    Determine the expected value of X.
  \end{enumerate}
\end{itemize}

\begin{Shaded}
\begin{Highlighting}[]
\NormalTok{expectation }\OtherTok{\textless{}{-}} \FunctionTok{sum}\NormalTok{(vals}\SpecialCharTok{*}\NormalTok{probs)}
\FunctionTok{print}\NormalTok{(expectation)}
\end{Highlighting}
\end{Shaded}

\begin{verbatim}
## [1] 4.6
\end{verbatim}

\hfill\break

\begin{itemize}
\item
  \begin{enumerate}
  \def\labelenumi{\alph{enumi})}
  \setcounter{enumi}{3}
  \tightlist
  \item
    Determine the variance of X.
  \end{enumerate}
\end{itemize}

\begin{Shaded}
\begin{Highlighting}[]
\NormalTok{squared\_diff }\OtherTok{\textless{}{-}}\NormalTok{ (vals }\SpecialCharTok{{-}}\NormalTok{ expectation)}\SpecialCharTok{\^{}}\DecValTok{2}
\NormalTok{variance }\OtherTok{\textless{}{-}} \FunctionTok{sum}\NormalTok{(squared\_diff }\SpecialCharTok{*}\NormalTok{ probs)}
\FunctionTok{print}\NormalTok{(variance)}
\end{Highlighting}
\end{Shaded}

\begin{verbatim}
## [1] 14.64
\end{verbatim}

\hfill\break

\begin{itemize}
\item
  \begin{enumerate}
  \def\labelenumi{\alph{enumi})}
  \setcounter{enumi}{4}
  \tightlist
  \item
    Determine the standard deviation of X.
  \end{enumerate}
\end{itemize}

\begin{Shaded}
\begin{Highlighting}[]
\NormalTok{stddev }\OtherTok{\textless{}{-}} \FunctionTok{sqrt}\NormalTok{(variance)}
\FunctionTok{print}\NormalTok{(stddev)}
\end{Highlighting}
\end{Shaded}

\begin{verbatim}
## [1] 3.826225
\end{verbatim}

\hfill\break

\hypertarget{problem-11}{%
\subsection{Problem 11}\label{problem-11}}

\hypertarget{problem-13}{%
\subsection{Problem 13}\label{problem-13}}

\hypertarget{problem-something}{%
\subsection{Problem something}\label{problem-something}}

\hypertarget{problem-something-2}{%
\subsection{Problem something 2}\label{problem-something-2}}

\end{document}
